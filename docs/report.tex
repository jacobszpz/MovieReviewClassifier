\documentclass[a4paper,12pt]{article}

% Code
\usepackage{minted}

% Referencing and more
\usepackage[british]{babel}
\usepackage{csquotes}
\usepackage[style=apa,backend=biber]{biblatex}


% Title page
\title{Sentiment Analysis of Movie Reviews}
\author{Jacob Sánchez}
\date{} % delete this line to display the current date


% .bib Files
\addbibresource{acl2011.bib}
\begin{document}

\maketitle

\section{Introduction}

The following report details the design, development, and testing of a sentiment analysis model which purpose is to classify movie reviews from a dataset depending on whether the reviewer's attitude towards the film was positive or negative. Reviews for training and testing were obtained from a dataset taken from IMDb (Internet Movie Database) \parencite{maas2011ACL}.

\section{Background Reading}
\section{Data}

The primary source of data for the training of this model is the Large Movie Review Dataset \parencite{maas2011ACL}. It contains 25,000 \enquote{highly polar} reviews for training purposes, and just as many additional reviews for testing.

The dataset was downloaded and the {\tt .tar.gz} file was unpacked. The dataset consists of two top-level folders named {\tt /train} and {\tt /test}, each with {\tt /pos} and {\tt /neg} subfolders, containing positive and negative reviews, respectively. Inside of these reside the reviews, each on one file, with a naming format of {\tt [\$id\_\$rating.txt]}, where id is a unique integer and rating is a score from 1 to 10 given along the review.

Before using the data, some preprocessing is necessary, since the reviews still contain some HTML tags.

\section{Model Development}
\section{Model Evaluation}
\section{Conclusion}

\printbibliography

\end{document}
